\section{Conclusiones provisorias}
Si bien se realizaron varias mediciones en distintos puntos de la ciudad de Mar del Plata, todavía falta cubrir diferentes zonas y diferentes horarios para poder realizar una conclusión certera acerca de la situación de radiación no ionizante en el Partido de General Pueyrredón. No obstante, con estos resultados provisorios, se comprueba que, en los días, horarios y zonas donde se efectuaron las mediciones, se cumple la norma de RNI y pr ende el lugar en cuestión cumple con el límite de exposición. En ninguno de los casos tuvo que aplicarse el método de la promediación temporal. \\

Respecto a la estimación de la potencia de la antena, en referencia a las mediciones alrededor de alguna de estas, no se han observado resultados que lleguen a comprobar que realmente la potencia estimada corresponde a la potencia de transmisión real. Esto seguramente se debe a que, en realidad, las antenas (en su mayoría telefónicas) emiten de manera direccional a medida que se comunican con sus respectivos clientes. \\

Como trabajo posterior, se recomienda seguir con las mediciones a la altura del suelo y tratar de encontrar una mejor manera de cargar los datos ya que la subida de cada uno se realiza de manera manual. También se sugiere buscar permisos en los edificios para realizar mediciones lo más parecidas a la línea de vista. Esto debido a la atenuación que ocurre por el efecto del desvanecimiento.

\subsection*{Referencias}

\begin{itemize}
    \item https://www.enacom.gob.ar/multimedia/normativas/1995/Resolucion\%20202\_95\%20MS.pdf
    \item https://www.enacom.gob.ar/multimedia/normativas/2004/Resolucion\%203690\_04\%20CNC.pdf
    \item https://cdn.educ.ar/dinamico/UnidadHtml\_\_get\_\_49690e2a-e366-41d2-834e-593e89deae06/ \\
    rni.educ.ar/proteccion-salud/quien-controla-cumplimiento-normas.html
\end{itemize}






\section{Programa para el análisis de datos}



\begin{figure}[H]
\centering
\includegraphics[width=1\linewidth]{images/codigo_1.png}
\includegraphics[width=1\linewidth]{images/codigo_2.png}

\caption{Código Python.}
\label{7.2}
\end{figure}



\begin{figure}[H]
\centering
\includegraphics[width=1\linewidth]{images/codigo_3.png}
\includegraphics[width=1\linewidth]{images/codigo_4.png}
\includegraphics[width=1\linewidth]{images/codigo_5.png}

\caption{Código Python.}
\label{7.4}
\end{figure}



\begin{figure}[H]
\centering
\includegraphics[width=1\linewidth]{images/codigo_6.png}
\includegraphics[width=1\linewidth]{images/codigo_7.png}
\includegraphics[width=1\linewidth]{images/codigo_8.png}
\caption{Código Python.}
\label{7.7}
\end{figure}

