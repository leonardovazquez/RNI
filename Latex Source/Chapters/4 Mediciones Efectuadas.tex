\section{Mediciones efectuadas}
En total, se realizaron más de 200 mediciones de densidad de potencia de inmisión en distintos puntos de la ciudad de Mar del Plata, perteneciente al Partido de General Pueyrredón, entre las fechas de Julio - Septiembre, en la franja horaria $15:00$h- $17:00$h durante los días de semana. Además, se han observado más de 70 antenas transmisoras de diferentes sistemas de comunicación: telefonía celular, radio broadcasting, enlaces privados, entre otras. Hay que tener en cuenta que varias antenas transmisoras no han podido ser vistas, esto es debido a que la gran mayoría son instaladas en lo más alto de los edificios.\\

En la figura siguiente se observa un mapa de la ciudad, en el cual se localizan las mediciones realizadas (verde a la altura del suelo y violeta a diferentes alturas) y la posición de las antenas (en azul). \\

\begin{figure}[H]
\centering
\includegraphics[width=0.6\linewidth]{images/mapa_1.png}
\caption{Mapa de la ciudad de Mar del Plata.}
\label{4.1}
\end{figure}

\begin{figure}[H]
\centering
\includegraphics[width=0.6\linewidth]{images/ampliacion.png}
\caption{Mapa de la ciudad de Mar del Plata.}
\label{4.12}
\end{figure}

Respecto al anterior, el siguiente mapa de calor muestra los puntos de medición más altos medidos desde el suelo. A pesar del color rojizo, no representan valores que superan el límite de $1000$ $mW/cm^2$.

\begin{figure}[H]
\centering
\includegraphics[width=0.6\linewidth]{images/mapa_2.png}
\caption{Mapa de calor de la ciudad de Mar del Plata.}
\label{4.2}
\end{figure}

En particular, se han realizado mediciones en las cercanías de algunas antenas: 

\begin{figure}[H]
\centering
\includegraphics[width=0.6\linewidth]{images/mapa_3.png}
\caption{Mediciones alrededor de antenas.}
\label{4.3}
\end{figure}


Por último, se realizaron mediciones a gran altura sobre algunos edificios: 

\begin{figure}[H]
\centering
\includegraphics[width=0.3\linewidth]{images/2_ant.jpeg}
\caption{Medición a 30 metros, frente a una antena de transmisión.}
\label{4.4}
\end{figure}

\begin{figure}[H]
\centering
\includegraphics[width=0.3\linewidth]{images/2.jpeg}
\caption{Medición a 30 metros, frente a una antena de transmisión.}
\label{4.5}
\end{figure}


\begin{figure}[H]
\centering
\includegraphics[width=0.8\linewidth]{images/vista.jpeg}
\caption{Medición a 60 metros, frente a varias antenas a línea de vista.}
\label{4.6}
\end{figure}

\begin{figure}[H]
\centering
\includegraphics[width=0.3\linewidth]{images/14.jpeg}
\caption{Medición a 60 metros, frente a varias antenas a línea de vista.}
\label{4.7}
\end{figure}
