\section{Introducción}

\subsection{Radiación Electromagnética}
Las ondas electromagnéticas están clasificadas de acuerdo a la frecuencia en la que operan distintos servicios y sistemas radioeléctricos. Dicha clasificación es determinada a nivel internacional por el organismo "Unión Internacional de Telecomunicaciones" (UIT), dependiente de la ONU (Organización de las Naciones Unidas). La razón de esto es para no interferir las comunicaciones. \\

Las radiaciones electromagnéticas de origen artificial proceden de diversos artefactos eléctricos/electrónicos creados por el hombre. Estas no poseen la energía suficiente para ionizar la materia (conversión de átomos de moléculas
en iones con carga eléctrica), es decir, demasiado débil para romper enlaces atómicos, razón por la cual se las denomina "Radiación No Ionizante" (RNI).

\begin{figure}[H]
\centering
\includegraphics[width=0.9\linewidth]{images/espectro.png}
\caption{Espectro electromagnético.}
\label{1.0}
\end{figure}

En diversos estudios se evaluaron los riesgos para la salud provocados por la exposición de RNI. Dichos trabajos llevaron a que se establecieran normas para asegurar la protección del ser humano y el medio ambiente.

\subsection{Antenas}
Las antenas son los elementos que permiten la irradiación y propagación de las ondas Electromagnéticas. Existen diversos tipos y tamaños, en función de la frecuencia y el sistema que se emplee para la comunicación: direccional, omnidireccional, parabólica, etc.\\

La estructura de soporte de las antenas no tiene relación con la potencia que estas irradian. En la figura que sigue se muestra la estructura de un sistema de telefonía celular.

\begin{figure}[H]
\centering
\includegraphics[width=0.7\linewidth]{images/cellular_antena.jpg}
\caption{Antena de telefonía celular.}
\label{1.1}
\end{figure}

La densidad superficial de potencia se define como la potencia por unidad de superficie en un determinado sitio. En el caso de las ondas electromagnéticas (particularmente radioeléctricas) se utiliza la unidad milésima de Watt ($mW$/$cm^2$). La potencia que irradia una antena no es directamente proporcional a la distancia a la que se esté midiendo, si no que disminuye con el cuadrado de la distancia.


\subsection{Normas de Protección}
Existen varios estándares por cada país que recomiendan los límites de exposición ocupacional y poblacional, expresados en densidad de potencia equivalente ($mW/cm^2$) en función de la frecuencia. En Argentina, el marco normativo sobre los niveles de la Máxima Exposición Poblacional (MEP) a las RNI están basados en las recomendaciones de la Organización Mundial de la Salud (OMS).\\

Con el objetivo de asegurar que la exposición humana a los campos electromagnéticos no tenga efectos perjudiciales para la salud y que los equipos generadores sean inocuos, se han adoptado diversas normas internacionales. La OMS basa sus recomendaciones a partir de la Comisión Internacional para la Protección contra las Radiaciones No Ionizantes (ICNIRP). \\

Hasta la actualidad, según la OMS, no existen evidencias científicas que permitan afirmar que las RNI produzcan efectos adversos sobre la salud de la población. El único efecto comprobado es que, cuando se sobrepasan ciertos límites, los tejidos empiezan a calentarse pero desaparece un tiempo después de quitar la fuente de radiación. En particular, respecto a la telefonía celular, la ICNIRP no indica ninguna necesidad respecto a algún tipo de precaución para el uso de teléfonos móviles o respecto a la instalación de las antenas de cobertura.\\

\subsubsection*{Resolución 202/1995}
El Ministerio de Salud y Acción social de la Nación establece los valores de Máxima Exposición Poblacional (MEP) para las Radiaciones No Ionizantes, los cuales están por debajo de lo que con posterioridad recomendó la OMS.

\subsubsection*{Resolución 530/2000}
La Secretaria de Comunicaciones adopta como norma los límites fijados por el Ministerio de Salud y dispone su aplicación obligatoria a todos los sistemas radioeléctricos.

\subsubsection*{Resolución 3690/2004}
La Comisión Nacional de Comunicaciones (CNC) establece el cumplimiento que deben observar los usuarios del espectro, respecto a diversos sistemas y/o servicios radioeléctricos sobre las RNI que emiten, los cuales deben ser acordes a los límites impuestos por el Ministerio de Salud Pública y adoptados posteriormente por la Secretaria de Comunicaciones. También se establece el Protocolo de Medición que se debe aplicar en todo el territorio nacional sobre las RNI, por parte de Técnicos o Profesionales que llevan adelante esta tarea. Asimismo, determina la excepción de efectuar mediciones en aquellos casos en los cuales la potencia emitida y la distancia de la población a los sistemas irradiantes así lo ameriten. 
